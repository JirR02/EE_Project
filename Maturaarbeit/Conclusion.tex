\section{Conclusion}

From the results of our research, one can derive two conclusions.\\
Firstly, one could observe that the piezoelectric element is limited by the maximum voltage output it can generate due to its thickness. This leads to the fact that the piezoelectric element will only produce its maximum voltage no matter how high the force is. Therefore, it is important to note that the formula is a theoretical approach to the voltage output of a piezoelectric element.\\
Finally, the experiment, the theoretical calculations, and the research papers showed that the power gained by the piezoelectric elements is not sufficient and not worth the costs as the revenue would not be profitable enough to cover the cost to implement piezoelectric elements. Furthermore, other sources producing renewable energy are far more profitable and generate more power. As for now, our technology is not sufficient for the use of piezoelectric elements as a source of renewable energy but perhaps in the near future, there will be a possibility where the piezoelectric element will function as a renewable energy source.\\
So, are piezoelectric elements a new renewable energy resource applicable in roads, railways, and pathways? Overall, the possibility of using piezoelectric elements as a renewable energy source is better understood. After simulating similar events where this concept could be used and understanding the limitations, one can now explain why this concept in our present will not be profitable enough to be used. Nevertheless, there are still many problems which have to be covered in this research.\\
The experiment is based on theoretical calculations due to the fact that the power produced is too small to be measured (Table \ref{Tab:Values of Power Graph 1}, \ref{Tab:Values of Power Graph 2}, \ref{Tab:Values of Power Graph 3}). In addition, the experiment is based on a very low frequency. This raises the question whether the theoretical calculations are applicable to the real results. Additionally, since the voltage output of the piezoelectric element used here is limited to 10 Volts, the experiment will never be able to provide the full potential of a piezoelectric element. This brings up the question whether the results from the experiment (Table \ref{Tab:Values of Voltage Graph 1}, \ref{Tab:Values of Voltage Graph 2}, \ref{Tab:Values of Voltage Graph 3}) are valid despite the theoretical calculated voltage having a discrepancy of $7.54 \pm 2.485 \%$ from the measured voltage. As we are limited to these materials, an experiment similar to the paper of M. Vázquez-Rodriguez, F. J. Jiménez, and J. De Frutos is not possible even though it would be more precise.\\
All in all, the research question could be answered. In our present time it is not worth implementing piezoelectric elements in roads, railways and pathways. Notwithstanding, its potential as a sensor is worth considering.