\chapter{Introduction}

\section{Overview}

Mankind is facing a challenge never encountered before. Researchers have shown that the globally rising temperatures are a consequence of us, humans, emitting green-house gases. Research shows that once we exceed global increase of 1.5 degrees, we will enter a feedback loop, where the overall temperature will not decrease. This has devastating effects on our ecosystem such as raising the water level of the oceans, extreme weather conditions as well as droughts and huge forest fires. These effects are already visible today. However, as much as humans cause global warming, humans are capable of preventing it. Over the past few years, countries around the world have sat together to negotiate goals to prevent climate change. Research was done to find alternatives to produce resources needed in our daily lives without emitting greenhouse gases. One of the biggest areas which is being researched now apart from sustainable transport is renewable energy. Aside from the commonly known solar panels and wind turbines, researchers are trying to find other ways to produce energy renewably and efficiently. Along with the ideas of bridges planted with carbon dioxide absorbing plants and roads made from solar panels, one idea caught our interest. The idea was to use piezoelectric elements in roads to harvest energy using the vibrations created by the vehicles. This idea could also be applied to railways and pedestrian walkways.\cite{Mumcu2021} \\
In this essay we are trying to answer the question, \textbf{“Are piezoelectric elements a new renewable energy resource applicable in roads, railways, and pathways?”}. The goal is to recreate a model of this concept and compare the measured output with the data provided from other sources. Furthermore, we are calculating the outcome of other studies and evaluate whether they are accurate. Lastly, we are deciding whether it is worth implementing this new energy resource and whether it has potential for the future. 

\subsection{Method}

To create a model which is as realistic as possible, we are conducting the experiment as follows. There are four piezo electric elements under the corners of a wooden board. To recreate the vibrations, a person will jump on the board. A voltmeter measured and graphed the voltage output of the experiment over a $470k\Omega$ resistor. This allowed us to measure the voltage created by the four piezoelectric elements as exact as possible. Moreover, we are calculating the theoretical voltage and energy output to have a comparison and a basis to verify the experiments from the other sources. Once all the data is compared, a conclusion can be drawn.

\subsection{Expectations}

Seeing that so many researchers praise this concept, the expectations of a high energy outcome are very high. Nevertheless, we expect a low outcome since the piezoelectric element can produce a high voltage but only for a short time which will reduce the energy. Furthermore, the concept is fairly new and not much research has been done regarding the energy outcome. There have been tests but they were never finished. We also expect that the piezoelectric element is more suitable in other areas rather than in energy harvesting and if it is used, it can only power small sensors. However, the concept could work, once the technology is provided.