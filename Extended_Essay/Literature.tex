\chapter{Literature}

The concept of implementing piezoelectric elements in roads, railways, and highways has been around for a while now. That is why it is apparent that there has been much research done about this concept. However, there has been much skepticism about this concept regarding many aspects such as the costs and the energy harvested by the piezoelectric elements. On the one hand, a scientist from the company khplant predicts that 12 meters of road could produce 1 kWh when a car drives over it. There is also a team from Lancaster University, who are very confident about this concept and are preparing to deploy tests in the United Kingdom and Italy. They believe that they could harvest 1 – 2 MW per kilometer road. This seems very doubtful as on the other hand, Rex Garland from Stanford University claims that 1 km of read contains 1.54 MJ which is miles away from the 1 kWh or the 1 – 2 MW. Nonetheless, how much energy could a busy road produce?\cite{khplant2023,Garland2013}\\
In average, an electric car needs $10kWh$ for $80 - 100 km$ which is about $0.9kWh$ per $km$. This means that $12m$ of road produce a maximum of $0.012kWh$ which is far away from the $1 kWh$ promissed by the scientist from the company khplant. On the other hand, the thesis of the teams of the Lancester University is thesible if it is considered that 47-93 cars drive over this road per day and there is no loss in energy when it is converted.\\  
\\
There are also papers which praise the idea of piezoelectric elements in roads as a way to harvest energy. M. Vázquez-Rodriguez, F. J. Jiménez, and J. De Frutos published a paper in 2011 where they measured the voltage and power output a public road could create using a contraption and different piezoelectric materials. They concluded that the power output depended on the speed of the cars and the frequency with which the cars drive over the piezoelectric element. Nevertheless, the voltage only peaked between 0.455 and 0.900 Volts with $70 kg$ of force pressing onto the piezoelectric elements and the wheels having a velocity of $60 - 115km/h$. In the end, the paper also showed the estimated power output of $48.3 \mu W$ using the formula $P = \frac{V^2}{R}$. \cite{M.VAZQUEZRODRIGUEZ2011}\\
In the paper by Hiba Najini and Arumugam Muthukumaraswamy published in 2017, they used a DC-DC Booster in addition to the capacitor, which would increase the power output. Theoretically, they could have an output for a 1 km road of 187 kWh considering that there are 3280 piezoelectric elements implemented into the road and that there were 500 cars per hour travelling at 100 km/h. They also included theoretical calculations including the time where the traffic density was lower than 500 cars per hour. However, the energy outcome is imense ranging from 20.8 to 93.81 kWh. compared to the $0.9kWh$ a car produces with out loss of energy during the harvesting process. \cite{Najini2017} \\
According to Flurin Solèr's research paper from 2017, it had a similar result as the experiment of this paper. In fact, he was also limited by the maximum voltage output of the piezoelectric element. He measured the voltage output and calculated the theoretical power output a person could generate while walking. Even though the concept of generating power during walking sounds promising, he was only able to generate an estimate of 0.247 mW when walking $7km/h$. He also calculated that it would take 6.2 years to charge an iPhone with this method.\cite{Soler2017}\\